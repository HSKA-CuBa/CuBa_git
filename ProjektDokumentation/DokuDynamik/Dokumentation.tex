\documentclass{article}
\usepackage[ngerman]{babel}
\usepackage[utf8]{inputenc}
\usepackage{graphicx} 
\usepackage{svg}
\usepackage{booktabs}
\usepackage{longtable, lscape}
\usepackage{tikz}
\usepackage{multicol}
\usepackage{longtable}
\usepackage{array} 
\usepackage{varwidth}
\graphicspath{{img/}}
\usepackage{geometry}
\usepackage{amsmath}
\usepackage{pdflscape}
\geometry{a4paper, top=25mm, left=30mm, right=25mm, bottom=20mm}

\begin{document}
\tableofcontents
\newpage

\section{Einführung}

Messungen:
\begin{itemize}
\item Beschleunigung Sensor1 und Sensor2 bei 45 Grad
\item Beschleunigung Sensor1 und Sensor2 bei 30 Grad
\item Beschleunigung Sensor1 und Sensor2 bei 15 Grad
\item Beschleunigung Sensor1 und Sensor2 bei 0  Grad
\item Beschleunigung Sensor1 und Sensor2 bei -15 Grad
\item Beschleunigung Sensor1 und Sensor2 bei -30 Grad
\item Beschleunigung Sensor1 und Sensor2 bei -45 Grad
\item Winkelgeschwindigkeit Sensor1 und Sensor2 bei 0 rad/sec
\end{itemize}

\section{Aktorik und Sensorik}
Der folgenden Abschnitt beschreibt die verwendeten elektrischen Bauteile, um einerseits die benötigten physikalischen Größen zu messen, und andererseits die verwendete Aktorik, um das Aufspringen und Balancieren der Würfelseite zu ermöglichen.
\newline 

Die Aufgabe der Sensorik besteht darin die Zustandsgrößen des Systemes zu bestimmen. Hierfür werden zwei \textit{GYR-521}-Platinen verwendet, die mit einem \textit{MPU6050}-IC der Firma \textit{InvenSense} bestückt sind. Diese bieten jeweils einen dreiachsigen Beschleunigungssensor und Gyroskop. Mit Hilfe dieser Messwerte können die Zustandsgrößen $\varphi$ und $\dot{\varphi}$ berechnet werden. Die Sensoren bieten die zusätzliche Möglichkeit einen variablen Tiefpassfilter zu verwenden um eine erste Glättung der Messwerte durchzuführen. Dieser Tiefpassfilter wird auf eine Grenzfrequenz von $44Hz$ eingestellt. Dieser Wert hat sich empirisch als optimaler Kompromiss zwischen Filterung der Rauschsignale und Verzögerung des eigentlichen Signals. Die Konfiguration und Auswertung der Sensoren erfolgt über eine $I^2C$-Schnittstelle. Die Justierung und Auswertung der Sensoren wird näher in \ref{sensorik_sec} beschrieben.
\newline

Abschnitt \ref{Dynamik_sec} zeigt den Einfluss eines Motormomentes auf die Position und Gewschwindigkeit der Würfelseite. Um diese Moment zu erzeugen wird ein bürstenloser DC-Motor der Firma \textit{MaxonMotor} verwendet (EC 45 flat, 50 Watt). Die Kriterien zur Auswahl des Motors sind einerseits die maximale Drehzahl und Drehmoment, andererseits die mechanische Zeitkonstante. Für das Aufspringen des Würfels ist die maximale Drehzahl des Motors von Bedeutung, die 10000 Umdrehung pro Minute des gewählten Motor reichen hierbei aus um eine ausreichend hohe kinetische Energie der Schwungmasse zu ermöglichen. Die Robustheit der Regelung wird durch das maximale Drehmoment limitiert, welches in diesem Fall bei 83.4 mNm liegt. Von besondere Bedeutung für die Regelung ist die mechanische Zeitkonstante des Motors, da diese eine Verzögerung der Stellgröße bewirkt und somit den geschlossenen Regelkreis negativ beeinflussen kann. Die mechanische Zeitkonstante des gewählten Motors ist mit $13.3ms$ im Vergleich zu anderen Kandidaten sehr niedrig. Die Ansteuerung des Motors erfolgt über den Treiberbaustein \textit{ESCON 36/3 EC}, welcher ebenfalls von der Firma \textit{Maxon Motor} vertrieben wird. Dieser ermöglicht die Steuerung des Drehmoments über ein PWM-Signal und die Auswertung der Winkelgeschwindigkeit $\dot{\psi}$ über ein analoges Signal.
\newline

Mit Hilfe einer mechanischen Bremse kann die Schwungmasse stoßartig zum Stillstand gebracht werden. Dadurch wird die kinetische Energie der Schwungmasse teilweise auf das Gesamtsystem übertragen und ermöglicht somit das Aufspringen. Die Bremsbacken werden über einen Servomotor betätigt, welcher mit Hilfe eines PWM-Signales kontrolliert wird.
\newline

Zur Ansteuerung der Aktorik und Sensorik wird ein STM32F4Discovery-Board der Firma \textit{STMicroelectronics} verwendet. Die Programmierung erfolgt über eine, auf Eclipse basierende, Toolkette. Um die Auswertung der Sensordaten und den Entwurf der Regelung zu erleichtern, wird der Quellcode anschließend in Simulink-Blöcke implementiert.


\section{Modellierung der Systemdynamik}
In dem folgenden Abschnitt werden die Bewegungsgleichungen mit Hilfe des Lagrange Formalismus hergeleitet. Aus diesen Gleichung kann im Anschluss eine Zustandsraumdarstellung aufgestellt werden, welche als Grundlage für den Reglerentwurf dient.

\begin{figure}[h]
\centering
\includegraphics[width=\linewidth]{MechZeichnung1D}
\caption{Mechanischer Aufbau, Quelle: eigene Darstellung}
\end{figure}

Der Prototyp besteht aus einem starren Körper der in $A$ auf einer Achse gelagert ist. In $B$ ist eine Schwungmasse über einen Motor mit dem Körper verbunden. Somit verfügt das Gesamtsystem über zwei Freiheitsgrade, welche durch die generalisierten Koordinaten 

\begin{equation}
q_1 = \varphi \hspace{35pt} q_2 = \psi
\end{equation}

beschrieben werden. Der Winkel $\varphi$ wird von den Achsen $y$ und $y_K$ eingeschlossen. Der Winkel beschreibt die rotatorische Verschiebung der Schwungmasse zu dem Körper. Die folgenden Größen beschreiben die weiteren physikalischen Gegebenheiten des Systems.\newline

\begin{table}[h]
\centering
\begin{tabular}{ll}
	$q_1 = \varphi$ & Ausfallwinkel des Körpers \\
	$q_2 = \psi$ & Winkel zwischen Schwungmasse und Körper \\
	$A$ & Drehpunkt des Körpers \\
	$B$ & Drehpunkt des Schwungrades \\
	$l_{AB}$ & Abstand zwischen $A$ und $B$ \\
	$l_{AC}$ & Abstand zwischen $A$ und dem Schwerpunkt des Körpers \\
	$m_K$ & Masse des Körpers \\
	$m_R$ & Masse des Schwungrades \\
	$J^A_K$ & Massenträgheitsmoment des Körper um $A$ \\
	$J^B_R$ & Massenträgheitsmoment der Schwungmasse um $B$ \\
	$C_{\varphi}$ & Dynamischer Reibkoeffizient des Körpers in $A$ \\
	$C_{\psi}$ & Dynamischer Reibkoeffizient des Schwungrades in $B$ \\
	$T_M$ & Drehmoment des Motor
\end{tabular}
\end{table}

\newpage
Um die Bewegungsgleichungen des Systems zu ermitteln wird der Lagrange Formalismus verwendet. Dieser basiert auf der Lagrange-Funktion $L$, welche die Differenz der kinetischen Energie $T$ und der potenziellen Energie $V$ des Systems beschreibt.

\begin{equation}
T = \frac{1}{2}[(J^A_K + m_R \cdot {l_{AB}}^2) {\dot{\varphi}}^2 + J^R_B(\dot{\varphi}+\dot{\psi})^2]
\end{equation}
\begin{equation}
V = g(m_R \cdot l_{AB} + m_K \cdot l_{AC})cos(\varphi)
\end{equation}
\begin{equation}
L = T - V = \frac{1}{2}[(J^A_K + m_R \cdot {l_{AB}}^2) {\dot{\varphi}}^2 + J^R_B(\dot{\varphi}+\dot{\psi})^2] - g(m_R \cdot l_{AB} + m_K \cdot l_{AC})cos(\varphi)
\end{equation}

Das von dem Motor verursachte Drehmoment $T_M$ verrichtet die virtuelle Arbeite $\delta W_M$. Aus diesem können die generalisierten Kraftkomponenten $Q_{\varphi}$ und $Q_{\psi}$ abgeleitet werden.

\begin{equation}
\delta W_M = T_M \cdot \delta \psi
\end{equation}
\begin{equation}
Q_{\varphi} = T_M \cdot \frac{\partial \psi}{\partial \varphi} = 0
\end{equation}
\begin{equation}
Q_{\psi} = T_M \cdot \frac{\partial \psi}{\partial \psi} = T_M
\end{equation}

Durch die Reibung in den Lagerungen bei $A$ und $B$ geht Energie in dem System verloren. Diese Verlustleistung kann mit den Rayleigh'schen Dissipationsfunktionen $D_{\varphi}$ und $D_{\psi}$ beschrieben werden.

\begin{equation}
D_{\varphi} = \frac{1}{2}C_{\varphi} \cdot {\dot{\varphi}}^2
\end{equation}
\begin{equation}
D_{\psi} = \frac{1}{2}C_{\psi} \cdot {\dot{\psi}}^2
\end{equation}
\begin{equation}
D = D_{\varphi} + D_{\psi} = \frac{1}{2}C_{\varphi} \cdot {\dot{\varphi}}^2 + \frac{1}{2}C_{\psi} \cdot {\dot{\psi}}^2
\end{equation}

Bei dem Prototyp handelt es sich um ein nicht konservatives System, da einerseits durch die Reibung mechanische Energie verloren geht. Andererseits erhöht das Motormoment die mechanische Gesamtenergie des Systems. Da die beiden generalisierten Koordinaten $\varphi$ und $\psi$ voneinander unabhängig sind können aus dem d'Alembert'schen Prinzip zwei Bewegungsgleichungen abgeleitet werden.

\begin{equation}
\frac{d}{dt}\frac{\partial L}{\partial \dot{q}_i}-\frac{\partial L}{\partial q_i} + \frac{\partial D}{\partial \dot{q}_i} = Q_i
\end{equation}
\begin{equation}
\frac{d}{dt}\frac{\partial L}{\partial \dot{\varphi}}-\frac{\partial L}{\partial \varphi} + \frac{\partial D}{\partial \dot{\varphi}} = Q_{\varphi} 
\end{equation}
\begin{equation}
\label{LG_phi_equation}
(J^A_K + J^R_B + m_R \cdot l_{AB})\ddot{\varphi} + J^R_B \cdot \ddot{\psi} - g(m_R \cdot l_{AB} + m_K \cdot l_{AC})sin(\varphi) + C_{\psi} \cdot \dot{\psi} = 0
\end{equation}
\begin{equation}
\frac{d}{dt}\frac{\partial L}{\partial \dot{\psi}}-\frac{\partial L}{\partial \psi} + \frac{\partial D}{\partial \dot{\psi}} = Q_{\psi} 
\end{equation}
\begin{equation}
\label{LG_psi_euqation}
J^R_B \cdot \ddot{\psi} = T_M - C_{\psi} \cdot \dot{\psi} - J^R_B \cdot \ddot{\varphi}
\end{equation}

Durch Einsetzen von (\ref{LG_psi_euqation}) in (\ref{LG_phi_equation}) ergibt sich die folgende Bewegungsgleichung für die Würfelseite.

\begin{equation}
\label{BG_phi_quation}
\ddot{\varphi} = \frac{g(m_R \cdot l_{AB} + m_K \cdot l_{AC})sin(\varphi) - C_{\varphi} \cdot \dot{\varphi} + C_{\psi} \cdot \dot{\psi} - T_M}{J^A_K + m_R \cdot l_{AB}}
\end{equation}

Die Bewegungsgleichung für die Schwungmasse ergibt sich durch Einsetzen von (\ref{BG_phi_quation}) in (\ref{LG_psi_euqation}).

\begin{equation}
\label{BG_psi_equation}
\ddot{\psi} = \frac{(J^A_K + m_R \cdot l_{AB} + J^R_B)(T_M - C_{\psi} \cdot \dot{\psi})}{(J^A_K + m_R \cdot {l_{AB}}^2)J^R_B} + \frac{C_{\varphi} \cdot \dot{\varphi} - g(m_R \cdot l_{AB} + m_K \cdot l_{AC})sin(\varphi)}{J^A_K + m_R \cdot {l_{AB}}^2}
\end{equation}

\section{Sensorik}
Die Aufgabe der verwendeten Sensorik liegt darin die Werte für $\varphi$, und $\dot{\varphi}$ zu bestimmen. Hierfür wurden zwei MPU6050 IC's verwendet. Diese verfügen jeweils über einen Beschleunigungssensor und Gyroskop, welche Werte für drei Achsen ausgeben. Um die Konfiguration und Auswertung der Sensoren vorzunehmen, bieten diese eine $I^2C$ Schnittstelle an. Die Position und Ausrichtung der Sensoren ist in \ref{Position_Sensoren_pic} dargestellt.

\begin{figure}[h]
\label{Position_Sensoren_pic}
\includegraphics[width=\linewidth]{SensorZeichnung1D}
\caption{Position der Sensoren, Quelle: eigene Darstellung}
\end{figure}

\subsection{Berechnung der Winkelwerte}
Die Sensoren keine Wege bzw. Winkel. Somit muss der Winkel $\varphi$ berechnet werden. Die gemessenen Sensorwerte hängen von $r_{S1}$ bzw. $r_{S2}$ ab, welche den Abstand zwischen den Sensoren und dem Drehpunkt $A$ beschreiben. Zusätzlich beeinflussen neben dem Winkel $\varphi$ auch dessen beiden Ableitungen $\dot{\varphi}$ und $\ddot{\varphi}$ die Sensorausgabe. Allerdings lassen sich aus den Beschleunigungswerten der beiden Sensoren wie folgt der aktuelle Wert von $\varphi$ berechnen.

\begin{equation}
\ddot{S}_i = 
\begin{pmatrix}
\ddot{x}_i \\ \ddot{y}_i \\ \ddot{z}_i
\end{pmatrix} =
\begin{pmatrix}
r_{Si} \cdot \ddot{\varphi} + sin(\varphi) \cdot g \\
- r_{Si} \cdot \dot{\varphi}^2 - cos(\varphi) \cdot g \\
0
\end{pmatrix}
\hspace{35pt}
i \in [1;2]
\end{equation}

\begin{equation}
\alpha = \frac{r_{S1}}{r_{S2}}
\end{equation}

\begin{equation}
\ddot{x}_1 - \alpha \cdot \ddot{x}_2 = 
g(1 - \alpha)sin(\varphi)
\end{equation}
\begin{equation}
\ddot{y}_1 - \alpha \cdot \ddot{y}_2 = 
-g(1- \alpha)cos(\varphi)
\end{equation}

\begin{equation}
\frac{\ddot{x}_1 - \alpha \cdot \ddot{x}_2}{\ddot{y}_1 - \alpha \cdot \ddot{y}_2} = -tan(\varphi)
\end{equation}



\section{Modellbildung und Bestimmung der Systemgrößen}
Mit Hilfe der Bewegungsgleichungen aus Abschnitt \ref{Dynamik_sec} kann nun eine Zustandsraumdarstellung aufgestellt werden. Hierfür werden die nichtlinearen Terme entsprechend linearisiert. 

\begin{equation}
\textbf{x} = \begin{pmatrix}
\varphi \\ \dot{\varphi} \\ \dot{\psi}
\end{pmatrix}
\hspace{35pt}
\textbf{y} = \begin{pmatrix}
\varphi \\ \dot{\varphi} \\ \dot{\psi}
\end{pmatrix}
\hspace{35pt}
u = T_M
\end{equation}

\begin{equation}
\dot{\textbf{x}} = \textbf{A} \cdot \textbf{x} + \textbf{B} \cdot u
\end{equation}

\begin{equation}
\textbf{y} = \textbf{C} \cdot \textbf{x} + \textbf{D} \cdot u
\end{equation}

\begin{equation}
\begin{split}
\renewcommand*{\arraystretch}{1.7}
\textbf{A} = \begin{pmatrix}
0 & 1 & 0 \\
\frac{g(m_K \cdot l_{AC} + m_R \cdot l_{AB})}{J^A_K + m_R \cdot l_{AB}^2} &
\frac{-C_{\varphi}}{J^A_K + m_R \cdot l_{AB}^2} & 
\frac{C_{\psi}}{J^A_K + m_R \cdot l_{AB}^2} \\
\frac{-g(m_K \cdot l_{AC} + m_R \cdot l_{AB)}}{J^A_K + m_R \cdot l_{AB}^2} &
\frac{C_{\varphi}}{J^A_K + m_R \cdot l_{AB}^2} &
\frac{-C_{\psi}(J^A_K + J^B_R + m_R \cdot l_{AB}^2)}{J^B_R(J^A_K + m_R \cdot l_{AB}^2)}
\end{pmatrix} 
\\
\renewcommand*{\arraystretch}{1.7}
\textbf{B} = \begin{pmatrix}
0 \\ \frac{-1}{J^A_K + m_R \cdot l_{AB}^2} \\ \frac{J^A_K + J^B_R + m_R \cdot l_{AB}^2}{J^K_R(J^A_K + m_R \cdot l_{AB}^2}
\end{pmatrix}
\hspace{35 pt}
\textbf{C} = \begin{pmatrix}
1 & 1 & 1
\end{pmatrix}
\hspace{35pt}
\textbf{D} = \begin{pmatrix}
0
\end{pmatrix}
\end{split}
\end{equation}

\section{Reglerentwurf}
Mit Hilfe der Zustandsraumdarstellung kann über die Rückführung des Zustandvektors eine Regelung entworfen werden. Das folgende Blockschaltbild zeigt den Zusammenhang der Systemmatrizen und der Reglermatrix $\textbf{F}$, welche zur Berechnung der Stellgröße $u=T_M$ dient.

\begin{figure}[h]
\label{Regelkreis_pic}
\includegraphics[width=\linewidth, trim={0 6.5cm 0 3.5cm}, clip]{Regelkreis}
\caption{Blockschaltbild Regelkreis, Quelle: eigene Darstellung, Inhalt aus \cite{RT2}}
\end{figure}

Die Stellgröße $u$ wird von einem Mikrokontroller mit einer Abtatsperiod $T_a = 20ms$ berechnet. Folglich handelt es sich um eine digitale Regelung. Um das Verhalten des diskreten Systems zu beschreiben müssen die diskreten Systemmatrizen $\textbf{A}_d$, $\textbf{B}_d$, $\textbf{C}_d$ und $\textbf{D}_d$ berechnet werden. Hierfür gilt nach \cite{RT2}:

\begin{equation}
\textbf{S} = T_a \sum_{v=0}^{\infty} \textbf{A}^v \frac{T^v}{(v+1)!}
\end{equation}
\begin{equation}
\textbf{A}_d = \textbf{I} + \textbf{S} \cdot \textbf{A}
\end{equation}
\begin{equation}
\textbf{B}_d = \textbf{S} \cdot \textbf{B}
\end{equation}
\begin{equation}
\textbf{C}_d = \textbf{C}
\end{equation}
\begin{equation}
\textbf{D}_d = \textbf{D}
\end{equation}

Die Reglermatrix $\textbf{F}$ wird als optimaler Zustandsregler nach dem quadratischen Gütekriterium entworfen. Die diskrete Gütefunktion für dieses System lautet:

\begin{equation}
\label{costfunction_equation}
I = \sum_{k=1}^\infty \textbf{x}^T(k) \cdot \textbf{Q} \cdot \textbf{x}(k) + R\cdot u(k)^2
\end{equation}

Die Matrizen $\textbf{Q}$ und $\textbf{R}$ stellen Gewichtungen der Zustands- und Stellgrößen dar. Die Ausgangswerte dieser Matrizen werden mit der Faustformel nach (\cite{lqrnotes}) berechnet. Ggf. können die Werte anschließend angepasst werden um die Reglergüte weiter zu verbessern.

\begin{equation}
\textbf{Q} = \begin{pmatrix}
\frac{1}{(\varphi_{max})^2} & 0 & 0 \\
0 & \frac{1}{(\dot{\varphi}_{max})^2} & 0 \\
0 & 0 & \frac{1}{(\dot{\psi}_{max})^2} \\
\end{pmatrix}
\end{equation}
\begin{equation}
R = \begin{pmatrix}
\frac{1}{(T_{M,max})^2}
\end{pmatrix}
\end{equation}

Die Reglermatrix $\textbf{F}$ muss die Eigenschaft besitzen die Gütefunktion (\ref{costfunction_equation}) zu minimieren. Dieses Problem wird mit Hilfe von der Matlab-Funktion \textit{lqrd} numerisch gelöst.

\include{Aufspringen}
\newpage
\begin{thebibliography}{\hspace{0.5cm}}
	\bibitem{Cubli1D} Mohanarjah Gajamohan, Michael merz, Igor Thommen, Raffaello D'Andrea: The Cubli: A Cube that can Jump Up and Balance
	\bibitem{Cubli3D_LQR} Mohanarajah Gajamohan, Michael Muehlbach, Tobias Widmer, Raffaello D'Andrea: The Cubli: A Reaction Wheel Based 3D Inverted Pendulum
	\bibitem{Cubli3D_Nonlinear} Michael Muehlbach, Gajamohan Mohanarajah, Raffaello D'Andrea: Nonlinear Analysis and Control of a Reaction Wheel-based 3D Inverted Pendulum
	\bibitem{TheoPhysik1} Wolfgang Nolting: Grundkurs Theoretische Physik 1 - Klassische Mechanik
	\bibitem{TheoPhysik2} Wolfgang Nolting: Grundkurs Theoretische Physik 2 - Analytische Mechanik
	\bibitem{Kane} Thomas R. Kane: Dynamics - Theory and Applications
	\bibitem{PraxisDerDigSigVer} Fernando Puente Le\'on, Sebastian Bauer: Praxis der digitalen Signalverarbeitung
	\bibitem{SimTechLinearUndNichtlinSysteme} Josef Hoffmann, Franz Quint : Simulation technischer linearer und nichtlinearer Systeme mit MATLAB/Simulink
	\bibitem{SystemTheoStochPro} Herbert Schlitt: Systemtheorie für stochastische Prozesse
	\bibitem{SigVer_AnaDigSig} Marin Meyer: Signalverarbeitung - Analoge und digitale Signale, Systeme und Filter
	\bibitem{SuS} Ottmar Beucher: Signale und Systeme - Theorie, Simulation und Anwendung
	\bibitem{RT1} Heinz Unbehauen: Regelungstechnik 1 - Klassische Verfahren zur Analyse und Synthese linearer kontinuierlicher Regelsysteme
	\bibitem{RT2} Heinz Unbehauen: Regelungstechnik 2 - Zustandsregelungen, digitale und nichtlineare Regelsysteme
	\bibitem{lqrnotes} Joao P. Hespanha: Lecture notes on LQR/LQG controller design
	\bibitem{ML_Mitchell} Tom M. Mitchell: Machine Learning
	\bibitem{ML_Bishop} Christopher Bishop: Pattern Recognition and Machine Learning
\end{thebibliography}


\end{document}