\section{Modellbildung und Bestimmung der Systemgrößen}
Mit Hilfe der Bewegungsgleichungen aus Abschnitt \ref{Dynamik_sec} kann nun eine Zustandsraumdarstellung aufgestellt werden. Hierfür werden die nichtlinearen Terme entsprechend linearisiert. 

\begin{equation}
\textbf{x} = \begin{pmatrix}
\varphi \\ \dot{\varphi} \\ \dot{\psi}
\end{pmatrix}
\hspace{35pt}
\textbf{y} = \begin{pmatrix}
\varphi \\ \dot{\varphi} \\ \dot{\psi}
\end{pmatrix}
\hspace{35pt}
u = T_M
\end{equation}

\begin{equation}
\dot{\textbf{x}} = \textbf{A} \cdot \textbf{x} + \textbf{B} \cdot u
\end{equation}

\begin{equation}
\textbf{y} = \textbf{C} \cdot \textbf{x} + \textbf{D} \cdot u
\end{equation}

\begin{equation}
\begin{split}
\renewcommand*{\arraystretch}{1.7}
\textbf{A} = \begin{pmatrix}
0 & 1 & 0 \\
\frac{g(m_K \cdot l_{AC} + m_R \cdot l_{AB})}{J^A_K + m_R \cdot l_{AB}^2} &
\frac{-C_{\varphi}}{J^A_K + m_R \cdot l_{AB}^2} & 
\frac{C_{\psi}}{J^A_K + m_R \cdot l_{AB}^2} \\
\frac{-g(m_K \cdot l_{AC} + m_R \cdot l_{AB)}}{J^A_K + m_R \cdot l_{AB}^2} &
\frac{C_{\varphi}}{J^A_K + m_R \cdot l_{AB}^2} &
\frac{-C_{\psi}(J^A_K + J^B_R + m_R \cdot l_{AB}^2)}{J^B_R(J^A_K + m_R \cdot l_{AB}^2)}
\end{pmatrix} 
\\
\renewcommand*{\arraystretch}{1.7}
\textbf{B} = \begin{pmatrix}
0 \\ \frac{-1}{J^A_K + m_R \cdot l_{AB}^2} \\ \frac{J^A_K + J^B_R + m_R \cdot l_{AB}^2}{J^K_R(J^A_K + m_R \cdot l_{AB}^2}
\end{pmatrix}
\hspace{35 pt}
\textbf{C} = \begin{pmatrix}
1 & 1 & 1
\end{pmatrix}
\hspace{35pt}
\textbf{D} = \begin{pmatrix}
0
\end{pmatrix}
\end{split}
\end{equation}