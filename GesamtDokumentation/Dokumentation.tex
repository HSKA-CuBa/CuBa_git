\documentclass{article}
\usepackage[ngerman]{babel}
\usepackage[utf8]{inputenc}
\usepackage{graphicx} 
\usepackage{epstopdf}
\usepackage{svg}
\usepackage{svg}
\usepackage{booktabs}
\usepackage{longtable, lscape}
\usepackage{tikz}
\usepackage{multicol}
\usepackage{longtable}
\usepackage{array} 
\usepackage{tabularx}
\usepackage{varwidth}
\graphicspath{{img/}}
\usepackage{geometry}
\usepackage{amsmath,mathtools}
\usepackage{pdflscape}
\usepackage[hidelinks]{hyperref}
\hypersetup{
    colorlinks=false, %set true if you want colored links
    linktoc=all
}
\geometry{a4paper, top=25mm, left=30mm, right=25mm, bottom=20mm}
\usepackage{titlesec}
\titleclass{\subsubsubsection}{straight}[\subsection]

\newcounter{subsubsubsection}[subsubsection]
\renewcommand\thesubsubsubsection{\thesubsubsection.\arabic{subsubsubsection}}
\renewcommand\theparagraph{\thesubsubsubsection.\arabic{paragraph}} % optional; useful if paragraphs are to be numbered

\titleformat{\subsubsubsection}
  {\normalfont\normalsize\bfseries}{\thesubsubsubsection}{1em}{}
\titlespacing*{\subsubsubsection}
{0pt}{3.25ex plus 1ex minus .2ex}{1.5ex plus .2ex}

\makeatletter
\renewcommand\paragraph{\@startsection{paragraph}{5}{\z@}%
  {3.25ex \@plus1ex \@minus.2ex}%
  {-1em}%
  {\normalfont\normalsize\bfseries}}
\renewcommand\subparagraph{\@startsection{subparagraph}{6}{\parindent}%
  {3.25ex \@plus1ex \@minus .2ex}%
  {-1em}%
  {\normalfont\normalsize\bfseries}}
\def\toclevel@subsubsubsection{4}
\def\toclevel@paragraph{5}
\def\toclevel@paragraph{6}
\def\l@subsubsubsection{\@dottedtocline{4}{7em}{4em}}
\def\l@paragraph{\@dottedtocline{5}{10em}{5em}}
\def\l@subparagraph{\@dottedtocline{6}{14em}{6em}}
\makeatother

\setcounter{secnumdepth}{4}
\setcounter{tocdepth}{4}

\newcommand{\inW}{\prescript{W}{}}
\newcommand{\inI}{\prescript{I}{}}
\newcommand{\bS}[1]{\boldsymbol{#1}}
\newcommand{\BinW}[1]{\inW{}\boldsymbol{#1}}
\newcommand{\BinI}[1]{\inI{}\boldsymbol{#1}}

\begin{document}
\tableofcontents
\newpage

\section{Einführung}

Messungen:
\begin{itemize}
\item Beschleunigung Sensor1 und Sensor2 bei 45 Grad
\item Beschleunigung Sensor1 und Sensor2 bei 30 Grad
\item Beschleunigung Sensor1 und Sensor2 bei 15 Grad
\item Beschleunigung Sensor1 und Sensor2 bei 0  Grad
\item Beschleunigung Sensor1 und Sensor2 bei -15 Grad
\item Beschleunigung Sensor1 und Sensor2 bei -30 Grad
\item Beschleunigung Sensor1 und Sensor2 bei -45 Grad
\item Winkelgeschwindigkeit Sensor1 und Sensor2 bei 0 rad/sec
\end{itemize}

\section{1D-Prototyp}
In diesem Teil wird der erste Prototyp vorgestellt. Hierbei handelt es sich um eine einzelne Würfelseite, welche mit Hilfe einer Achse gelagert ist. Dadurch wird die Bewegung des Systems auf zwei rotatorische Freiheitsgrade beschränkt, nämlich die Rotation um die Achse und die Bewegung der Schwungmasse relativ zu der Würfelseite. Mit Hilfe dieses Entwurfes kann die Dynamik und Anforderungen an die Komponenten an einem vereinfachten Modell untersucht werden. Aus diesen Ergebnisse können dann Rückschlüsse auf den Entwurf des endgültigen Würfels gezogen werden.
\newline


\begin{figure}[h!]
\includegraphics[width=\linewidth]{img/1D_Model_pic.JPG}
\caption{1D-Modell, Quelle: eigene Darstellung}
\end{figure}
\include{1D_AktorikSensorik}
\subsection{Modellierung der Systemdynamik} \label{Dynamik_sec}
In dem folgenden Abschnitt werden die Bewegungsgleichungen mit Hilfe des Lagrange Formalismus hergeleitet. Aus diesen Gleichung kann im Anschluss eine Zustandsraumdarstellung aufgestellt werden, welche als Grundlage für den Reglerentwurf dient.

\begin{figure}[h]
\centering
\includegraphics[width=\linewidth]{MechZeichnung1D}
\caption{Mechanischer Aufbau, Quelle: eigene Darstellung}
\end{figure}

Der Prototyp besteht aus einem starren Körper der in $A$ auf einer Achse gelagert ist. In $B$ ist eine Schwungmasse über einen Motor mit dem Körper verbunden. Somit verfügt das Gesamtsystem über zwei Freiheitsgrade, welche durch die generalisierten Koordinaten 

\begin{equation}
q_1 = \varphi \hspace{35pt} q_2 = \psi
\end{equation}

beschrieben werden. Der Winkel $\varphi$ wird von den Achsen $y$ und $y_K$ eingeschlossen. Der Winkel beschreibt die rotatorische Verschiebung der Schwungmasse zu dem Körper. Die folgenden Größen beschreiben die weiteren physikalischen Gegebenheiten des Systems.\newline

\begin{table}[h]
\centering
\begin{tabular}{|c|c|}
\hline
	\textbf{Variable} & \textbf{Erklärung} \\ \hline
	$q_1 = \varphi$ & Ausfallwinkel des Körpers \\ \hline
	$q_2 = \psi$ & Winkel zwischen Schwungmasse und Körper \\ \hline
	$A$ & Drehpunkt des Körpers \\ \hline
	$B$ & Drehpunkt des Schwungrades \\ \hline
	$l_{AB}$ & Abstand zwischen $A$ und $B$ \\ \hline
	$l_{AC}$ & Abstand zwischen $A$ und dem Schwerpunkt des Körpers \\ \hline
	$m_K$ & Masse des Körpers \\ \hline
	$m_R$ & Masse des Schwungrades \\ \hline
	${\theta}^A_K$ & Massenträgheitsmoment des Körper um $A$ \\ \hline
	${\theta}^B_R$ & Massenträgheitsmoment der Schwungmasse um $B$ \\ \hline
	$C_{\varphi}$ & Dynamischer Reibkoeffizient des Körpers in $A$ \\ \hline
	$C_{\psi}$ & Dynamischer Reibkoeffizient des Schwungrades in $B$ \\ \hline
	$T_M$ & Drehmoment des Motor \\ \hline
\end{tabular}
\end{table}

\newpage
Um die Bewegungsgleichungen des Systems zu ermitteln wird der Lagrange Formalismus verwendet. Dieser basiert auf der Lagrange-Funktion $L$, welche die Differenz der kinetischen Energie $T$ und der potenziellen Energie $V$ des Systems beschreibt.

\begin{equation}
T = \frac{1}{2}[({\theta}^A_K + m_R \cdot {l_{AB}}^2) {\dot{\varphi}}^2 + {\theta}^B_R(\dot{\varphi}+\dot{\psi})^2]
\end{equation}
\begin{equation}
V = g(m_R \cdot l_{AB} + m_K \cdot l_{AC})cos(\varphi)
\end{equation}
\begin{equation}
L = T - V = \frac{1}{2}[({\theta}^A_K + m_R \cdot {l_{AB}}^2) {\dot{\varphi}}^2 + {\theta}^B_R(\dot{\varphi}+\dot{\psi})^2] - g(m_R \cdot l_{AB} + m_K \cdot l_{AC})cos(\varphi)
\end{equation}

In dem System wirken unterschiedliche Kräfte. Einerseits erzeugt der Motor ein Drehmoment, welches die virtuelle Arbeite $\delta W_M$ verursacht. Andererseits verrichtet die Gravitation die virtuelle Arbeite $\delta W_G$. Zusätzlich muss die, durch die Reibung entstandene, Verlustleistung berücksichtigt werden. In diesem Fall wird die Reibleistung mit den Rayleigh'schen Dissipationsfunktionen $D_{\varphi}$ und $D_{\psi}$ beschrieben und verrichten die virtuelle Arbeit $\delta W_D$.

\begin{equation}
-\delta W_M = T_M \cdot \delta \psi
\end{equation}

\begin{equation}
-\delta W_G = g(m_K \cdot l_{AC} + m_R \cdot l_{AB})sin(\varphi) \cdot \delta \varphi
\end{equation}

\begin{equation}
D_{\varphi} = \frac{1}{2}C_{\varphi} \cdot {\dot{\varphi}}^2
\end{equation}
\begin{equation}
D_{\psi} = \frac{1}{2}C_{\psi} \cdot {\dot{\psi}}^2
\end{equation}
\begin{equation}
D = D_{\varphi} + D_{\psi} = \frac{1}{2}C_{\varphi} \cdot {\dot{\varphi}}^2 + \frac{1}{2}C_{\psi} \cdot {\dot{\psi}}^2
\end{equation}
\begin{equation}
-\delta W_D = - C_{\varphi} \cdot \dot{\varphi} \cdot \delta \varphi - C_{\psi} \cdot \dot{\psi} \cdot \delta \psi
\end{equation}

Die Summe der virtuellen Arbeiten, welche von den verschiedenen Kräften verrichtet wird, ergibt die virtuelle Arbeit des Gesamtsystems $\delta W$. In dem die verrichtete Arbeit partiell nach den beiden generalisierten Koordinaten $\varphi$ und $\psi$ differenziert wird, können die beiden generalisierten Kraftkomponenten $Q_{\varphi}$ und $Q_{\psi}$ berechnet werden.

\begin{equation}
Q_{\varphi} = g(m_K \cdot l_{AC} + m_R \cdot l_{AB})sin(\varphi) - C_{\varphi} \cdot \dot{\varphi}
\end{equation}
\begin{equation}
Q_{\psi} = T_M - C_{\psi} \cdot \dot{\psi}
\end{equation}


Bei dem Prototyp handelt es sich um ein nicht konservatives System, da durch die Reibung mechanische Energie verloren geht und der Motor dem System mechanische Energie zuführt. Da die beiden generalisierten Koordinaten $\varphi$ und $\psi$ voneinander unabhängig sind können aus dem d'Alembert'schen Prinzip zwei Bewegungsgleichungen abgeleitet werden.

\begin{equation}
\frac{d}{dt}\frac{\partial T}{\partial \dot{q}_i}-\frac{\partial T}{\partial q_i} = Q_i
\end{equation}
\begin{equation}
\frac{d}{dt}\frac{\partial T}{\partial \dot{\varphi}}-\frac{\partial T}{\partial \varphi} = Q_{\varphi} 
\end{equation}
\begin{equation}
\label{LG_phi_equation}
({\theta}^A_K + {\theta}^B_R + m_R \cdot l_{AB}^2)\ddot{\varphi} + {\theta}^B_R \cdot \ddot{\psi} - g(m_R \cdot l_{AB} + m_K \cdot l_{AC})sin(\varphi) + C_{\varphi} \cdot \dot{\varphi} = 0
\end{equation}
\begin{equation}
\frac{d}{dt}\frac{\partial T}{\partial \dot{\psi}}-\frac{\partial T}{\partial \psi} } = Q_{\psi} 
\end{equation}
\begin{equation}
\label{LG_psi_euqation}
{\theta}^R_B \cdot \ddot{\psi} = T_M - C_{\psi} \cdot \dot{\psi} - {\theta}^B_R \cdot \ddot{\varphi}
\end{equation}

Durch Einsetzen von (\ref{LG_psi_euqation}) in (\ref{LG_phi_equation}) ergibt sich die folgende Bewegungsgleichung für die Würfelseite.

\begin{equation}
\label{BG_phi_quation}
\ddot{\varphi} = \frac{g(m_R \cdot l_{AB} + m_K \cdot l_{AC})sin(\varphi) - C_{\varphi} \cdot \dot{\varphi} + C_{\psi} \cdot \dot{\psi} - T_M}{{\theta}^A_K + m_R \cdot l_{AB}^2}
\end{equation}

Die Bewegungsgleichung für die Schwungmasse ergibt sich durch Einsetzen von (\ref{BG_phi_quation}) in (\ref{LG_psi_euqation}).

\begin{equation}
\label{BG_psi_equation}
\ddot{\psi} = \frac{({\theta}^A_K + m_R \cdot l_{AB}^2 + {\theta}^B_R)(T_M - C_{\psi} \cdot \dot{\psi})}{({\theta}^A_K + m_R \cdot {l_{AB}}^2){\theta}^B_R} + \frac{C_{\varphi} \cdot \dot{\varphi} - g(m_R \cdot l_{AB} + m_K \cdot l_{AC})sin(\varphi)}{{\theta}^A_K + m_R \cdot {l_{AB}}^2}
\end{equation}
\subsection{Sensorik}
\label{Sensorik_sec}
Die Aufgabe der verwendeten Sensorik liegt darin die Werte für $\varphi$, und $\dot{\varphi}$ zu bestimmen. Hierfür wurden zwei MPU6050 IC's verwendet. Diese verfügen jeweils über einen Beschleunigungssensor und Gyroskop, welche Werte für drei Achsen ausgeben. Der Tiefpass der Sensoren wird auf eine Grenzfrequenz von $44Hz$ eingestellt, da hier einerseits eine erste Glättung der Daten erfolgt, andererseits aber keine zu große Verzögerung ergibt, welche sich wiederum negativ auf die Regelung auswirken könnte. Die Position und Ausrichtung der Sensoren ist in \ref{Position_Sensoren_pic} dargestellt.

\begin{figure}[h]
\includegraphics[width=\linewidth]{SensorZeichnung1D}
\caption{Position der Sensoren, Quelle: eigene Darstellung}

\label{Position_Sensoren_pic}
\end{figure}

\subsubsection{Winkelschätzung}
Die Sensoren sind nicht in der Lage Wege bzw. Winkel zu messen. Somit muss der Winkel $\varphi$ berechnet werden. Die gemessenen Beschleunigungen setzten sich aus einem statischen Anteil, welcher von $\varphi$ abhängt, und einem dynamischen Anteil, welcher von $\dot{\varphi}$ bzw. $\ddot{\varphi}$ abhängt, zusammen.

\begin{equation}
\ddot{S}_i = 
\begin{pmatrix}
\ddot{x}_i \\ \ddot{y}_i \\ \ddot{z}_i
\end{pmatrix} =
\begin{pmatrix}
r_{Si} \cdot \ddot{\varphi} + sin(\varphi) \cdot g \\
- r_{Si} \cdot \dot{\varphi}^2 - cos(\varphi) \cdot g \\
0
\end{pmatrix}
\hspace{35pt}
i \in [1;2]
\end{equation}

Da die dynamischen Anteile zusätzlich von dem Abstand $r_{Si}$ abhängen, kann die geometrische Anordnung der beiden Sensoren genutzt werden um das Verhältnis der beiden Anteile zu berechnen. Somit kann der aktuelle Wert von $\varphi$ nach \cite{Cubli1D} wie folgt berechnen.
\begin{equation}
\alpha = \frac{r_{S1}}{r_{S2}}
\end{equation}

\begin{equation}
\ddot{x}_1 - \alpha \cdot \ddot{x}_2 = 
g(1 - \alpha)sin(\varphi)
\end{equation}
\begin{equation}
\ddot{y}_1 - \alpha \cdot \ddot{y}_2 = 
-g(1- \alpha)cos(\varphi)
\end{equation}

\begin{equation}
\frac{\ddot{x}_1 - \alpha \cdot \ddot{x}_2}{\ddot{y}_1 - \alpha \cdot \ddot{y}_2} = -tan(\varphi)
\end{equation}

Nach dem selben Prinzip kann auch die Winkelbeschleunigung $\ddot{\varphi}$ berechnet werden.
\begin{equation}
\ddot{x}_1 - \ddot{x}_2 = [r_{S1} \cdot \ddot{\varphi} + sin(\varphi) \cdot g] - [r_{S2} \cdot \ddot{\varphi} + sin(\varphi) \cdot g] = (r_{S1} - r_{S2}) \cdot \ddot{\varphi}
\end{equation}
\begin{equation}
\ddot{\varphi} = \frac{\ddot{x}_1 - \ddot{x}_2}{r_{S1} - r_{S2}}
\end{equation}

\subsubsection{Kalibrierung und Justierung}
Die Sensoren geben die Beschleunigungs- und Geschwindigkeitswerte als 16 Bit Werte im Zweierkomplement aus. Diese Rohwerte müssen in die mit Hilfe eines Ausgleichspolynoms in die jeweilige SI-Einheit umgerechnet werden. 

\subsubsubsection{Umrechnung der Beschleunigungswerte}
Um das Polynom zur Umrechnung der Beschleunigungswerte zu ermitteln werden sieben Messungen in den fixen Ausfallpositionen $\phi \in [-45, -30, -15, 0, 15, 30, 45]$ durchgeführt. Pro Position werden $m = 10000$ Messwerte aufgenommen. Da in der Ruhelage die Beschleunigung lediglich von dem aktuellen Ausfallwinkel abhängt ist der Sollwert für jede Position bekannt. Somit kann ein Polynom erster Ordnung approximiert werden um Mittelwerte der sieben Positionen in die entsprechenden Beschleunigungswerte umzurechnen.

\begin{table}[h]
\centering
\begin{tabular}{lcllcl}
$\ddot{x}_n$ &$\equiv$& X-Beschleunigung Sensor n &
$\ddot{x}^R_n$ &$\equiv$& X-Rohwert Sensor n \\
$\ddot{y}_n$ &$\equiv$& Y-Beschleunigung Sensor n &
$\ddot{y}^R_n$ &$\equiv$& Y-Rohwert Sensor n
\end{tabular}
\end{table}

\vspace*{-\baselineskip}
\begin{equation}
\ddot{x}_n = p^1_{x_n} \cdot \ddot{x}^R_n + p^2_{x_n} \hspace{35pt} \vert \hspace{3pt} n \in \{1, 2\}
\end{equation}
\begin{equation}
\ddot{y}_n = p^1_{y_n} \cdot \ddot{y}^R_n + p^2_{y_n} \hspace{35pt} \vert \hspace{3pt} n \in \{1, 2\}
\end{equation}
\vspace*{-\baselineskip}
\begin{table}[h]
\centering
\begin{tabular}{lcllcl}
$p^1_{x_1}$ &$=$& $-5.992 \cdot 10^{-4}$ & $p^2_{x_1}$ &$=$& $0.3328$ \\
$p^1_{x_2}$ &$=$& $-6.003 \cdot 10^{-4}$ & $p^2_{x_2}$ &$=$& $0.4138$ \\
$p^1_{y_1}$ &$=$& $-6.127 \cdot 10^{-4}$ & $p^2_{y_1}$ &$=$& $0.1186$ \\
$p^1_{y_2}$ &$=$& $-6.81 \cdot 10^{-4}$ & $p^2_{y_2}$ &$=$& $0.1143$ \\
\end{tabular}
\end{table}

\vspace*{-\baselineskip}
\begin{figure}[h]
	\includegraphics[width=0.5\linewidth]{img/X1__dd___fitted.eps}
	\includegraphics[width=0.5\linewidth]{img/X2__dd___fitted.eps}
\end{figure}

\vspace*{-\baselineskip}
\begin{figure}[h!]
	\includegraphics[width=0.5\linewidth]{img/Y1__dd___fitted.eps}
	\includegraphics[width=0.5\linewidth]{img/Y2__dd___fitted.eps}
\end{figure}

\subsubsubsection{Umrechnung der Winkelgeschwindigkeiten}
Um die Rohwerte der Gyroskope in Winkelgeschwindigkeiten umzurechnen wird die Würfelseite fixiert und die Winkelgeschwindigkeitswerte der beiden Sensoren aufgenommen. Hierbei werden jeweils $m = 1000$ Werte aufgenommen. Da der Sollwert $\dot{\varphi} = 0 \frac{m}{s}$ bekannt ist kann die systematische Messabweichung der Sensoren über den Mittelwert bestimmt werden. Der proportionale Umrechnungsfaktor von Rohdaten zu Winkelgeschwindigkeiten wird dem Datenblatt des Herstellers entnommen.
\begin{figure}[h]
	\includegraphics[width=0.5\linewidth]{img/phi1__d.eps}
	\includegraphics[width=0.5\linewidth]{img/phi2__d.eps}
\end{figure}
\vspace*{-\baselineskip}
\begin{table}[h]
\centering
\begin{tabular}{lcllcl}
$\dot{\varphi}_n$ & $\equiv$ & $\varphi$-Geschwindigkeit Sensor n & $\dot{\varphi}^R_n$ & $\equiv$ $\dot{\varphi}$-Rohwert Sensor n
\end{tabular}
\end{table}
\vspace*{-\baselineskip}
\begin{equation}
\dot{\varphi}_n = p^1_{\dot{\varphi}^R_n}  \cdot (\dot{\varphi}_n + p^2_{\dot{\varphi}_n})
\end{equation}
\vspace*{-\baselineskip}
\begin{table}[h]
\centering
\begin{tabular}{lcllcl}
$p^1_{\varphi_1}$ &$=$& $-1.3265 \cdot 10^{-4}$ & $p^2_{\varphi_1}$ &$=$& $441.3160$ \\
$p^1_{\varphi_2}$ &$=$& $-1.3265 \cdot 10^{-4}$ & $p^2_{\varphi_2}$ &$=$& $76.5140$ \\
\end{tabular}
\end{table}
\vspace*{-\baselineskip}
\subsubsection{Auswertung der Radgeschwindigkeit $\dot{\psi}$}
Der Motortreiber liefert ein analoges Spannungssignal, welches die aktuelle Motorgeschwindigkeit wiedergibt. Um die ADC-Werte in SI-Einheiten umzurechnen wird ein Polynom erster Ordnung benötigt. Hierfür werden mit Hilfe der ESCON-Studio konstante Motorgeschwindigkeiten ($\dot{\psi} \in \{ -3000, -2000,$  $-1000, 0, 1000, 2000, 3000 \} [rpm] $) gefahren und pro Durchlauf $m=500$ ADC-Werte aufgenommen. Über die Mittelwerte der Messungen und die vorgegebenen Radgeschwindigkeiten wird anschließend ein Polynom erster Ordnung approximiert.
\begin{table}[h!]
\centering
\begin{tabular}{lcllcl}
$\dot{\psi}$ & $\equiv$ & Geschwindigkeit der Schwungmasse & $\dot{\psi}_{ADC}$ & $\equiv$ & ADC-Wert
\end{tabular}
\end{table}
\vspace*{-\baselineskip}
\begin{equation}
\dot{\psi} = -0.5092 \cdot \dot{\psi}_{ADC} + 1050
\end{equation}
\vspace*{-\baselineskip}
\begin{figure}[h!]
\centering
	\includegraphics[width=0.5\linewidth]{img/ADC_mittelwert_polynom.eps}
\end{figure}

\subsection{Modellbildung und Bestimmung der Systemgrößen}
Mit Hilfe der Bewegungsgleichungen aus Abschnitt \ref{Dynamik_sec} kann nun eine Zustandsraumdarstellung aufgestellt werden. Hierfür werden die nichtlinearen Terme entsprechend linearisiert. Mit Hilfe der Bewegungsgleichungen bzw. Zustandsraumdarstellung kann ein Simulink-Modell implementiert werden um das Systemverhalten zu simulieren. Mit Hilfe der Zustandsraumdarstellung wird ein Zustandsregler entworfen, welcher an dem Modell erprobt werden kann. Zusätzlich über die Simulation der Einfluss der einzelnen Parameter, Sensorrauschen und Störungen untersucht werden.

\begin{equation}
\textbf{x} = \begin{pmatrix}
\varphi \\ \dot{\varphi} \\ \dot{\psi}
\end{pmatrix}
\hspace{35pt}
\textbf{y} = \begin{pmatrix}
\varphi \\ \dot{\varphi} \\ \dot{\psi}
\end{pmatrix}
\hspace{35pt}
u = T_M
\end{equation}

\begin{equation}
\dot{\textbf{x}} = \textbf{A} \cdot \textbf{x} + \textbf{B} \cdot u
\end{equation}
\begin{equation}
\textbf{y} = \textbf{C} \cdot \textbf{x} + \textbf{D} \cdot u
\end{equation}
\begin{equation}
\setlength{\jot}{10pt}
\begin{split}
\renewcommand*{\arraystretch}{1.7}
\textbf{A} = \begin{pmatrix}
0 & 1 & 0 \\
\frac{g(m_K \cdot l_{AC} + m_R \cdot l_{AB})}{{\theta}^A_K + m_R \cdot l_{AB}^2} &
\frac{-C_{\varphi}}{{\theta}^A_K + m_R \cdot l_{AB}^2} & 
\frac{C_{\psi}}{{\theta}^A_K + m_R \cdot l_{AB}^2} \\
\frac{-g(m_K \cdot l_{AC} + m_R \cdot l_{AB)}}{{\theta}^A_K + m_R \cdot l_{AB}^2} &
\frac{C_{\varphi}}{{\theta}^A_K + m_R \cdot l_{AB}^2} &
\frac{-C_{\psi}({\theta}^A_K + {\theta}^B_R + m_R \cdot l_{AB}^2)}{{\theta}^B_R({\theta}^A_K + m_R \cdot l_{AB}^2)}
\end{pmatrix} 
\\
\renewcommand*{\arraystretch}{1.7}
\textbf{B} = \begin{pmatrix}
0 \\ \frac{-1}{{\theta}^A_K + m_R \cdot l_{AB}^2} \\ \frac{{\theta}^A_K + {\theta}^B_R + m_R \cdot l_{AB}^2}{{\theta}^K_R({\theta}^A_K + m_R \cdot l_{AB}^2}
\end{pmatrix}
\hspace{35 pt}
\textbf{C} = \begin{pmatrix}
1 & 1 & 1
\end{pmatrix}
\hspace{35pt}
\textbf{D} = \begin{pmatrix}
0
\end{pmatrix}
\end{split}
\end{equation}

\subsubsection{Identifikation der Parameter}
Der Reglerentwurf und die Simulation erfordern eine möglichst präzise Bestimmung der Systemparameter, wie z.B. Längen, Massen, Massenträgheitsmomente und Reibwerte. Die Bestimmung der Längen $l_{AB}$ und $l_{AC}$, der Massen $m_K$, $m_R$ und $m_G$, der Massenträgheitsmomente $\theta^A_K$ und $\theta^B_R$ erfolgt über das CAD-Modell. Hierfür werden Bauteile mit einer nicht homogenen Massenverteilung, wie z.B. die Motoren, in separate Baugruppen mit homogener Massenverteilung unterteilt.

\subsubsubsection{Ermittlung des Reibwertes $C_{\varphi}$}
In dem die Schwungmasse fest mit der Würfelseite verbunden wird ergibt sich die folgende Bewegungsgleichung für das Gesamtsystem.

\begin{equation}
\label{ermittlung_c_phi_equation}
(\theta^A_K + \theta^B_R + m_R  \cdot l_{AB}^2) \ddot{\varphi} = g(m_K \cdot l_{AC} + m_R \cdot l_{AB})sin(\varphi) - C_{\varphi} \cdot \dot{\varphi}
\end{equation}

In dem Versuchsaufbau wird das Gesamtsystem nun von einem Startwinkel $\varphi_0$ losgelassen, woraufhin eine gedämpfte Schwingung entsteht. Mit Hilfe der Sensoren können die Größen $\varphi$, $\dot{\varphi}$ und $\ddot{\varphi}$ gemessen werden.

\begin{figure}[h!]
\centering
 \includegraphics[width=0.5\linewidth]{img/C_phi.eps}
	\caption{Ausfallwinkel der Würfelseite bei Versuch 4, Quelle: eigene Darstellung}
\end{figure}

Über die $n$ Messpunkte ergeben sich die folgenden Vektoren.

\begin{equation}
\boldsymbol{\varphi} = \begin{pmatrix} \varphi_1 \\ \varphi_2 \\ \vdots \\ \varphi_n \end{pmatrix} \hspace{35pt}
\boldsymbol{\dot{\varphi}} = \begin{pmatrix}
\dot{\varphi_1} \\ \dot{\varphi_2} \\ \vdots \\ \dot{\varphi_n}
\end{pmatrix} \hspace{35pt}
\boldsymbol{\ddot{\varphi}} = \begin{pmatrix}
\ddot{\varphi_1} \\ \ddot{\varphi_2} \\ \vdots \\ \ddot{\varphi_n}
\end{pmatrix}
\end{equation}

Damit ergibt sich durch Umstellen von \ref{ermittlung_c_phi_equation} die folgende Gleichung.

\begin{equation}
C_{\varphi} \cdot \boldsymbol{\dot{\varphi}} = g(m_K \cdot l_{AC} + m_R \cdot l_{AB})sin(\boldsymbol{\varphi}) - (\theta^A_K + \theta^B_R + m_R  \cdot l_{AB}^2) \boldsymbol{\ddot{\varphi}}
\end{equation}

Mit Hilfe der Methode der kleinsten Fehlerquadrate kann nun der Reibwert $C_{\varphi}$ bestimmt werden.

\begin{equation}
C_{\varphi} = 5.4 \cdot 10^{-3} \cdot kg \cdot m^2 \cdot s^{-1}
\end{equation}


\subsubsubsection{Ermittlung des Reibwertes $C_{\psi}$}
Im nächsten Versuchsaufbau wird die Würfelseite fixiert ($\dot{\varphi} = 0$). Hierbei beschleunigt der Motor die Schwungmasse mit einem konstanten Drehmoment $T_M=10mNm$. $T_M$ ist so zu wählen, dass sich die Radgeschwindigkeit $\dot{\psi}$ in einem Bereich bewegt, welcher dem Arbeitsbereich des geschlossenen Regelkreises entspricht. 

\begin{figure}[h!]
\centering
\includegraphics[width=0.6\linewidth]{img/C_psi.eps}
\caption{Versuch 5: Verlauf der Radgeschwindigkeit, Quelle: eigene Darstellung}
\end{figure}


Da die Bewegung auf einen Freiheitsgrad beschränkt wurde vereinfacht sich das Modell des Systems auf die folgende Bewegungsgleichung.

\begin{equation}
\label{ermittlung_c_psi_equation}
\theta^B_R \cdot \ddot{\psi} = T_M - C_\psi \cdot \dot{\psi}
\end{equation}

Im Versuchsverlauf werden bei $n$ Stützstellen die Werte von $\psi$, $\dot{\psi}$ und $\ddot{\psi}$ gemessen. Daraus ergeben sich die folgenden Vektoren.

\begin{equation}
\label{ermittlung_c_psi_vektoren_equation}
\boldsymbol{\psi} = \begin{pmatrix} \psi_1 \\ \psi_2 \\ \vdots \\ \psi_n \end{pmatrix} \hspace{35pt}
\boldsymbol{\dot{\psi}} = \begin{pmatrix}
\dot{\psi_1} \\ \dot{\psi_2} \\ \vdots \\ \dot{\psi_n}
\end{pmatrix} \hspace{35pt}
\boldsymbol{\ddot{\psi}} = \begin{pmatrix}
\ddot{\psi_1} \\ \ddot{\psi_2} \\ \vdots \\ \ddot{\psi_n}
\end{pmatrix}
\end{equation}

Durch Einsetzen von \ref{ermittlung_c_psi_vektoren_equation} in \ref{ermittlung_c_psi_equation} kann über die Methode der kleinsten Fehlerquadrate wiederum der Reibwert $C_\psi$ bestimmt werden.

\begin{equation}
C_{\psi}= 4.8301 \cdot 10^{-6} \cdot kg \cdot m^2 \cdot s^{-1}
\end{equation}

\subsubsubsection{Resultate der Systemidentifikation}
An Hand der beschriebenen Versuche und Methoden wurden die folgenden Werte für die Parameter des Gesamtsystems ermittelt.

\begin{table}[h]
\centering
\begin{tabular}{|c|c|}
	\hline
	\textbf{Parameter} & \textbf{Wert} \\ \hline
	$l_{AB}$ & $0.084m$\\ \hline
	$l_{AC}$ & $0.087m$ \\ \hline
	$m_K$ & $0.221kg$ \\ \hline
	$m_R$ & $0.09kg$ \\ \hline
	${\theta}^A_K$ & $2.8 \cdot 10^{-3}kg \cdot m^2$ \\ \hline
	${\theta}^B_R$ & $1.1683 \cdot 10^{e-4} \cdot kg \cdot m^2$ \\ \hline
	$C_{\varphi}$ & $6.2 \cdot 10^{-3} \cdot kg \cdot m^2 \cdot s^{-1}$ \\ \hline
	$C_{\psi}$ & $3.1176 \cdot 10^{-5} \cdot kg \cdot m^2 \cdot s^{-1}$ \\ \hline
	$r_{S1}$ & $0.14m$ \\ \hline
	$r_{S2}$ & $0.061m$ \\ \hline
\end{tabular}
\end{table}

\newpage
\subsubsection{Entwurf des Simulink-Modelles}
Dieser Abschnitt erklärt den Aufbau des Simulink-Modelles zur Simulation des Systems. Die oberste Modellschicht besteht aus drei Subsystemen zur Simulation des Motor, der Würfelseite und der Schwungmasse.

\begin{figure}[h]
\label{Simulink_1DModell_Overview}
\includegraphics[width=\linewidth, trim={0 7cm 0 6cm},clip]{model_1D_overview}
\caption{Simulink-Modell Übersicht, Quelle: eigene Darstellung}
\end{figure}

\subsubsubsection{Simulation des Motors}
Der Motor wird als zwei in Reihe geschaltete PT1-Glieder simuliert. Da der Regler als Stellgröße ein Motormoment berechnet, beträgt die Verstärkung des Motor $K_M$ in der Simulation den Wert eins. Die Zeitkonstanten der PT1-Glieder sind einerseits die elektrische Zeitkonstante $T_e$ und die mechanische Zeitkonstante $T_m$, wessen Werte dem Datenblatt des Herstellers entnommen werden.

\begin{equation}
K_M = 1 \hspace{35pt} T_e = 0.55ms \hspace{35pt} T_m = 12.4ms
\end{equation}

\subsubsubsection{Simulation der Würfelseite}
Die Dynamik der Würfelseite wird von \ref{BG_phi_quation} beschrieben.

\begin{equation}
\ddot{\varphi} = \frac{g(m_R \cdot l_{AB}^2 + m_K \cdot l_{AC})sin(\varphi) - C_{\varphi} \cdot \dot{\varphi} + C_{\psi} \cdot \dot{\psi} - T_M}{{\theta}^A_K + m_R \cdot l_{AB}^2} \tag{\ref{BG_phi_quation}}
\end{equation}

Somit ist die Winkelbeschleunigung gleich der Summe der Drehmomente geteilt durch die betroffenen Massenträgheitsmomente. Durch Integration und Rückführung können die einzelnen Drehmomente berechnet werden. Das folgende Modell zeigt die Umsetzung dieser Berechnungsvorschrift in Simulink.

\begin{figure}[h]
\label{Simulink_1DModell_CubeBody_pic}
\includegraphics[width=\linewidth, trim={0 5cm 0 5cm},clip]{model_1D_cubebody}
\caption{Subsystem Würfelseite, Quelle: eigene Darstellung}
\end{figure}

\subsubsubsection{Simulation der Schwungmasse}
Die Dynamik der Schwungmasse wird von \ref{BG_psi_equation} beschrieben, allerdings wird das Modell vereinfacht indem $\ddot{phi}$ nicht substituiert wird.

\begin{equation}
{\theta}^R_B \cdot \ddot{\psi} = T_M - C_{\psi} \cdot \dot{\psi} - {\theta}^B_R \cdot \ddot{\varphi}\tag{\ref{BG_psi_equation}}
\end{equation}

Das Simulink-Modell folgt dem selben Schema wie das Subsystem zur Simulation der Bewegung des Würfelkörpers.

\begin{figure}[h]
\label{Simulink_1DModell_Wheel_pic}
\includegraphics[width=\linewidth, trim={0 6cm 0 6cm},clip]{model_1D_wheel}
\caption{Subsystem Schwungmasse, Quelle: eigene Darstellung}
\end{figure}
\subsection{Reglerentwurf}
In dem folgenden Abschnitt wird der Entwurf eines Reglers vorgestellt, welcher auf der Rückführung des Zustandsvektors basiert. Im ersten Teil wird die analytische Bestimmung der Parameter erläutert, daraufhin wird der Regler im zweiten Teil an dem Prototyp erprobt und validiert.

\subsubsection{Analytische Bestimmung der Reglerparameter}
Mit Hilfe der Zustandsraumdarstellung kann über die Rückführung des Zustandvektors eine Regelung entworfen werden. Das folgende Blockschaltbild zeigt den Zusammenhang der Systemmatrizen und der Reglermatrix $\textbf{F}$, welche zur Berechnung der Stellgröße $u=T_M$ dient.

\begin{figure}[h]
\label{Regelkreis_pic}
\includegraphics[width=\linewidth, trim={0 6.5cm 0 3.5cm}, clip]{Bilder_RT}
\caption{Blockschaltbild Regelkreis, Quelle: eigene Darstellung, Inhalt aus \cite{RT2}}
\end{figure}

Die Stellgröße $u$ wird von einem Mikrokontroller mit einer Abtastperiod $T_a = 20ms$ berechnet. Folglich handelt es sich um eine digitale Regelung. Um das Verhalten des diskreten Systems zu beschreiben müssen die diskreten Systemmatrizen $\textbf{A}_d$, $\textbf{B}_d$, $\textbf{C}_d$ und $\textbf{D}_d$ berechnet werden. Hierfür gilt nach \cite{RT2}:

\begin{equation}
\textbf{S} = T_a \sum_{v=0}^{\infty} \textbf{A}^v \frac{T^v}{(v+1)!}
\end{equation}
\begin{equation}
\textbf{A}_d = \textbf{I} + \textbf{S} \cdot \textbf{A}
\end{equation}
\begin{equation}
\textbf{B}_d = \textbf{S} \cdot \textbf{B}
\end{equation}
\begin{equation}
\textbf{C}_d = \textbf{C}
\end{equation}
\begin{equation}
\textbf{D}_d = \textbf{D}
\end{equation}

Die Reglermatrix $\textbf{F}$ wird als optimaler Zustandsregler nach dem quadratischen Gütekriterium entworfen. Die diskrete Gütefunktion für dieses System lautet:

\begin{equation}
\label{costfunction_equation}
I = \sum_{k=1}^\infty \textbf{x}^T(k) \cdot \textbf{Q} \cdot \textbf{x}(k) + R\cdot u(k)^2
\end{equation}

Die Matrizen $\textbf{Q}$ und $\textbf{R}$ stellen Gewichtungen der Zustands- und Stellgrößen dar. Die Ausgangswerte dieser Matrizen werden mit der Faustformel nach (\cite{lqrnotes}) berechnet. Ggf. können die Werte anschließend angepasst werden um die Reglergüte weiter zu verbessern.

\begin{equation}
\textbf{Q} = \begin{pmatrix}
\frac{1}{(\varphi_{max})^2} & 0 & 0 \\
0 & \frac{1}{(\dot{\varphi}_{max})^2} & 0 \\
0 & 0 & \frac{1}{(\dot{\psi}_{max})^2} \\
\end{pmatrix}
\end{equation}
\begin{equation}
R = \begin{pmatrix}
\frac{1}{(T_{M,max})^2}
\end{pmatrix}
\end{equation}

Die Reglermatrix $\textbf{F}$ muss die Eigenschaft besitzen die Gütefunktion (\ref{costfunction_equation}) zu minimieren. Dieses Problem wird mit Hilfe von der Matlab-Funktion \textit{lqrd} numerisch gelöst.

\subsubsection{Verifizierung des Reglers an dem 1D-Prototyp}
Mit Hilfe von Matlab wurden die Werte der Reglermatrix $\boldsymbol{F}$ berechnet.

\begin{equation}
\boldsymbol{F} = \begin{pmatrix}
0.8821 & 0.1386 & 0.0002
\end{pmatrix}
\end{equation}

Somit lässt sich das Motormoment $T_{M,n}$ durch die Rückführung des Zustandvektors $\boldsymbol{x}_n$ über die Reglermatrix $\boldsymbol{F}$ berechnen.

\begin{equation}
T_{M,n} = \boldsymbol{F} \cdot \boldsymbol{x}_n
\end{equation}

Der Regler wird zuerst mit Hilfe eines Simulink-Modelles in der Simulation überprüft. Anschließend wird der geschlossene Regelkreis auf den Prototyp übertragen. Hierbei ist zu beachten, dass bei der Modellierung des Systemverhaltens die Annahme getroffen wurde, dass der Schwerpunkt des Systems auf dessen Y-Achse liegt. Durch den unsymmetrischen Aufbau ist dies allerdings nicht der Fall, somit ergibt sich das folgende Gravitationsmoment  $M_G$, wobei der Winkel $\varphi_{cog}$ den Winkel zwischen Y-Achse und Schwerpunkt der Würfelseite bezeichnet.

\begin{equation}
M_G = g(m_K \cdot l_{AC} + m_R \cdot l_{AB}) \cdot sin(\varphi + \varphi_{cog})
\end{equation}

Somit muss ein konstantes Motormoment erzeugt werden, um die Würfelseite bei dem Sollwinkel $\varphi = 0$ zu halten. Dadurch wird die Schwungmasse konstant beschleunigt, weshalb die Schwungmasse nicht vollständig zum Stillstand kommen kann. Deshalb muss für die Berechnung des Drehmomentes der Winkel $\varphi_{cog}$ zu der Zustandsgröße $\varphi$ addiert werden.

\begin{equation}
T_{M,n} = \boldsymbol{F} \cdot \begin{pmatrix}
\varphi + \varphi_{cog} \\
\dot{\varphi} \\
\dot{\psi}
\end{pmatrix}
\end{equation}

Da die Winkelgeschwindigkeit der Schwungmasse $\dot{\psi}$ nur dann verschwindet wenn das Motormoment gleich null ist, kann der Wert von $\varphi_{cog}$ empirisch ermittelt werden.


\subsection{Aufspringen}
Das Aufspringen der Würfelseite wird durch das abrupte bremsen der Schwungmasse ermöglicht. Hierbei wird der Drehimpuls der Schwungmasse auf das Gesamtsystem übertragen. Dieser Vorgang kann als nicht elastischer Stoß modelliert werden. Somit ergibt sich aus dem Drehimpulserhaltungssatz folgende Gleichung, wobei $\dot{\varphi}_B$ die Winkelgeschwindigkeit der Würfelseite nach dem Bremsen und $\dot{\psi}_B$ die Winkelgeschwindigkeit der Schwungmasse vor dem Bremsen darstellt.

\begin{equation}
\label{impulserhaltung_bremsen_equation}
(\theta^A_K + \theta^B_R + m_R \cdot l_{AB}) \cdot \dot{\varphi}_B = \theta^B_R \cdot \dot{\psi}_B
\end{equation} 

Um die Würfelseite von der Ruhelage ($\varphi_R = \pm \frac{\pi}{4}$) zu dem Gleichgewichtspunkt ($\varphi_G = 0$) zu bewegen muss Arbeite verrichtet werden. Diese Arbeit $W$ ist gleich der Änderung der potentiellen Energie von der Ruhelage hin zu dem Gleichgewichtspunkt.

\begin{equation}
\label{arbeit_bremsen_equation}
W = V(\varphi_G) - V(\varphi_R) = g(m_K + m_R) l_{AC} \cdot (cos(\varphi_G) - cos(\varphi_R))
\end{equation}

Auf Grund des Energieerhaltungssatzes muss die kinetische Energie des Gesamtsystemes nach dem Bremsvorgang gleich der zu leistenden Arbeit sein um die Würfelseite aufzurichten. Somit kann der Zusammenhang von $\dot{\varphi}_B$ und der Arbeit $W$ wie folgt beschrieben werden.

\begin{equation}
\label{energie_bremsen_equation}
\frac{1}{2}(\theta^A_K + \theta^B_R + m_R \cdot l_{AB}) \dot{\varphi}_B^2 = g(m_K + m_R) l_{AC} \cdot (1 - \frac{1}{\sqrt{2}})
\end{equation}

Mit Hilfe der Gleichungen (\ref{energie_bremsen_equation}) und (\ref{impulserhaltung_bremsen_equation}) kann nun die notwendige Bremsgeschwindigkeit $\dot{\psi}_B$ berechnet werden.

\begin{equation}
\label{psi_bremsen_equation}
\dot{\psi}_B = \sqrt{(2-\sqrt{2}(m_R + m_K) \cdot l_{AC} \cdot g \cdot \frac{\theta^A_K + \theta^B_R + m_R \cdot l_{AB}}{{\theta^B_R}^2}}
\end{equation}

Das obige Modell geht von der Annahme aus, dass es sich um einen perfekt nicht elastischen Stoß handelt und bei der Bewegung der Würfelseite keine Energie verloren geht. Somit besteht eine Abweichung des Modells von den realen Bedingungen. Um diese Abweichungen zu minimieren wird ein, an den Gradientenabstieg angelehnter, Lernalgorithmus implementiert. Nach dem Abbremsen der Schwungmasse werden die Größen $\varphi$ und $\dot{\varphi}$ beobachtet. Tritt ein Nulldurchgang von $\varphi$ auf bedeutet dies, dass die Anfangsgeschwindigkeit $\dot{\varphi}_B$ und somit die Radgeschwindigkeit $\dot{\psi}_B$ zu hoch waren. Tritt jedoch ein Nulldurchgang von $\dot{\varphi}$ auf, folgt, dass $\dot{\varphi}_B$ und $\dot{\psi}_B$ zu niedrig waren. In beiden Fällen kann die Änderung der Energie $\Delta E$, welche nötig ist um den Zielpunkt zu erreichen, berechnet werden.

\begin{equation}
\Delta E = \begin{cases}
\begin{matrix}
(1-cos(\varphi_0))(m_K+m_r)l_{AC} \cdot g  & \hspace{35pt} \vert \hspace{5pt} \dot{\varphi} = 0 \\
-\frac{1}{2}(\theta^A_K + \theta^B_R + m_R \cdot l_{AB}) \cdot \dot{\varphi}^2_0 & \hspace{35pt} \vert \hspace{5pt} {\varphi} = 0 \\
\end{matrix}

\end{cases}
\end{equation}

Mit Hilfe der Drehimpuls- (\ref{impulserhaltung_bremsen_equation}) und Energieerhaltung (\ref{energie_bremsen_equation} wird nun aus der Energieänderung $\Delta E$ die nötige Änderung der Radgeschwindigkeit $\Delta \dot{\psi}_B$ berechnet.

\begin{equation}
\pm \Delta \dot{\psi}_B = \sqrt{2 \cdot \frac{\theta^A_K + \theta^B_R + m_R \cdot l_{AB}}{{\theta^B_R}^2} \cdot \Delta E}
\end{equation}

Die Konvergenz des Lernalgorithmus gegen den Zielwert wird empirisch bewiesen, hierfür ist allerdings das hinzufügen einer Lernrate $\mu$ erforderlich. Daraus ergibt sich letztendlich folgende Vorschrift um den aktuellen Wert der Bremsgeschwindigkeit $\dot{\psi}_B$ zu bestimmen.

\begin{equation}
\dot{\psi}_B := \dot{\psi}_B + \mu \cdot \Delta \dot{\psi}_B \hspace{35pt} \vert \hspace{5pt} 0 < \mu \le 1
\end{equation}

\section{3D-Modell}
Der zweite Teil des Projektes beschäftigt sich mit der Entwicklung des vollständigen Würfels (3D-Modell). Hierbei können die Ansätze des 1D-Modell übernommen werden, wobei deren Komplexität jedoch zunimmt. Beispielsweise werden aus den zwei Freiheitsgraden der Würfelseite sechs Freiheitsgrade für das 3D-Modell. Folglich steigt der Umfang der Systemanalyse und des Regelkreises. Eine besondere Schwierigkeit besteht darin, dass die drei Eingangsgrößen alle Systemzustände beeinflussen und somit nicht getrennt betrachtet werden können. Auch die Schätzung der Zustände mit Hilfe der Sensorwerte nimmt zu da sich die Anzahl der Sensorsignale verdreifacht und komplexere Beziehungen zwischen den Sensorwerten und den Systemzuständen bestehen.

Die Vorgehensweise ist dennoch identisch zu dem 1D-Modell, allerdings wird gezielt auf die Unterschiede zwischen den beiden Modellen eingegangen und Ergebnisse aus vorhergehenden Analyse als bekannt angenommen.

\begin{figure}[h!]
\includegraphics[width=\linewidth]{img/3D_Modell_img.JPG}
\caption{3D-Modell, Quelle: eigene Darstellung}
\end{figure}
\subsection{Modellierung der Systemdynamik}
In diesem Abschnitt werden die Bewegungsgleichungen des 3D-Modells mit Hilfe des Lagrange-Formalismus hergeleitet. Der Würfelkörper besitzt drei rotatorische Freiheitsgrade, die drei Schwungmassen verfügen über jeweils einen Freiheitsgrad. Somit ergeben sich insgesamt sechs Freiheitsgrade für das Gesamtsystem. Dadurch steigt die Komplexität der Systemdynamik stark an, allerdings bestehen nach wie vor Parallelen zu der Dynamik des 1D-Modells.
\newline

Um die Position des Würfels zu beschreiben wird ein raumfestes Bezugssystem $\{I\}$ eingeführt, welches von den drei Einheitsvektoren $\inI e_x$, $\inI e_y$ und $\inI e_z$ beschrieben wird. Das zweite Bezugssystem $\{W\}$ ist körperfest und rotiert somit mit dem Würfel. Es wird von den Einheitsvektoren $\inW e_x$, $\inW e_y$ und $\inW e_z$ beschrieben.

\begin{figure}[h]
\centering
\includegraphics[width=\linewidth]{MechZeichnung3D}
\caption{Mechanischer Aufbau, Quelle: eigene Darstellung}
\end{figure}

Die aktuelle Position des Würfels kann somit eindeutig durch die Verschiebung des körperfesten Bezugssystems $\{W\}$ zu dem raumfesten Bezugssystem $\{I\}$ bestimmt werden. Um diese Rotation zu beschreiben werden die drei Euler-Winkel $\varphi_1$, $\varphi_2$ und $\varphi_3$ eingeführt. Die Drehreihenfolge und Drehachsen werden im folgenden beschrieben.

\begin{table}[h]
\centering
\begin{tabular}{|c|c|}
\hline
\textbf{Winkel} & \textbf{Beschreibung} \\ \hline
$\varphi_1$ & Drehung um $\inI e_z$ \\ \hline
$\varphi_2$ & Drehung um $\inW e_x$  \\ \hline
$\varphi_3$ & Drehung um $\inW e_z$ \\ \hline
\end{tabular}
\end{table}

Mit Hilfe der Euler-Winkel können Drehmatrizen definiert werden um Koordinaten in dem raumfesten Bezugssystem $\{I\}$ in das körperfeste Bezugssystem $\{W\}$ zu projizieren. Hier zeigt sich wieder die Bedeutung der Reihenfolge der einzelnen Drehungen, da auch die Matrizenmultiplikation im Allgemeinen nicht kommutativ ist.

\begin{equation}
\inW{}\boldsymbol{r} = \inW{}\bS{D}_{\varphi_1} \cdot \inW{}\bS{D}_{\varphi_2} \cdot \inW{}\bS{D}_{\varphi_3} \cdot \inI{}\bS{r} = \inW{}\bS{D} \cdot \BinI{r}
\end{equation}
\begin{equation}
\inW{}\bS{D}_{\varphi_1} = \begin{pmatrix}
c_{\varphi_1} & -s_{\varphi_1} & 0 \\
s_{\varphi_1} & c_{\varphi_1} & 0 \\
0 & 0 & 1
\end{pmatrix}
\hspace{5pt}
\inW{}\bS{D}_{\varphi_2} = \begin{pmatrix}
1 & 0 & 0 \\
0 & c_{\varphi_2} & -s_{\varphi_2} \\
0 & s_{\varphi_2} & c_{\varphi_2}
\end{pmatrix}
\hspace{5pt}
\inW{}\bS{D}_{\varphi_3} = \begin{pmatrix}
c_{\varphi_3} & -s_{\varphi_3} & 0  \\
s_{\varphi_3} & c_{\varphi_3} & 0 \\
0 & 0 & 1
\end{pmatrix}
\end{equation}
\begin{equation}
\BinW{D} = \begin{pmatrix}
c_{\varphi_1}c{\varphi_3} - s_{\varphi_1}c_{\varphi_2}s_{\varphi_3} &
-c_{\varphi_1}s_{\varphi_3} - s_{\varphi_1}c_{\varphi_2}c_{\varphi_3} &
s_{\varphi_1}s_{\varphi_2} \\
s_{\varphi_1}c_{\varphi_3}+c_{\varphi_1}c_{\varphi_2}s_{\varphi_3} &
-s_{\varphi_1}s_{\varphi_3}+c_{\varphi_1}c_{\varphi_2}{\varphi_3} &
-c_{\varphi_1}s_{\varphi_2} \\
s_{\varphi_2}s_{\varphi_3} &
s_{\varphi_2}c_{\varphi_3} &
c_{\varphi_2}
\end{pmatrix}
\end{equation}

Die Projizierung einer Koordinate aus dem körperfesten in das raumfeste Bezugssystem erfolgt durch die transponierte der Matrix $\BinW{D}$.

\begin{equation}
\BinI{r} = \BinI{D} \cdot \BinI{r} = \BinW{D}^T \cdot  \BinI{r}
\end{equation} 

Die Bewegung der Schwungmassen relativ zu dem Würfelkörper wird von den drei Winkeln $\psi_1$, $\psi_2$ und $\psi_3$ beschrieben. Deren zeitliche Ableitungen stellen die Winkelgeschwindigkeiten der Schwungräder dar. 

\begin{equation}
\bS{\psi} = \begin{pmatrix}
{\psi}_1 \\
{\psi}_2  \\
{\psi}_3 
\end{pmatrix}
\hspace{35pt}
\bS{\dot{\psi}} = \begin{pmatrix}
\dot{\psi}_1 \\
\dot{\psi}_2  \\
\dot{\psi}_3 
\end{pmatrix}
\end{equation}

\subsubsection{Potential des Systems}
Um die Lagrange-Funktion $L$ des Systems zu bestimmen muss einerseits die kinetische Energie $T$ und die potentielle Energie $V$ ermittelt werden. Das Potential des Würfels wird durch die aktuelle Lage seines Schwerpunktes $\bS{r}$ bestimmt, hierbei ist lediglich die z-Komponente des raumfesten Bezugssystem von Bedeutung.

\begin{equation}
V = m_G \cdot g \cdot \inI z_{cog}
\end{equation}

Die Position des Schwerpunktes im körperfesten Bezugssystem ist fix. Durch die Projektion dieses Vektors $\BinW{r}_{cog}$ in das raumfeste Bezugssystem $\{I\}$ wird die Abhängigkeit von der aktuellen Verschiebung berücksichtigt.

\begin{equation}
\BinI{r}_C = \BinI{D} \cdot \BinW{r}_{C} = \BinI{D} \cdot \begin{pmatrix}
\inW x_C \\ \inW y_C \\ \inW z_C
\end{pmatrix}
\end{equation}

Folglich ergibt sich der folgende Zusammenhang für das Potential $V$ und die aktuelle Ausrichtung des Würfels.

\begin{equation}
V = m_G \cdot g \cdot \inI z_{cog} = m_G \cdot g \cdot (s_{\varphi_2}s_{\varphi3} \cdot \inW x_C + s_{\varphi_2}c_{\varphi_3} \cdot \inW y_C  + c_{\varphi_2} \cdot \inW z_C)
\end{equation}

\subsubsection{Kinetische Energie des Systems}
Die kinetische Energie setzt sich aus der Winkelgeschwindigkeit des Würfels $\bS{\omega}_K$ und der Geschwindigkeiten der drei Schwungmassen $\bS{\omega}_R$ zusammen. Hierbei ist zu beachten, dass die Winkelgeschwindigkeiten in verschiedenen Bezugssystemen darstellbar sind und die kinetische Energie von der Darstellungsform unabhängig ist. Um dies zu gewährleisten müssen allerdings auch die Trägheitstensoren in das jeweilige Bezugssystem projiziert werden.

\begin{equation}
\BinW{\omega}_K = \begin{pmatrix}
\inW \omega_x \\
\inW \omega_y \\
\inW \omega_z
\end{pmatrix}
\hspace{35pt}
\BinW{\omega}_R = \dot{\bS{\psi}}
\end{equation}

\begin{equation}
T = \frac{1}{2} \BinW{\omega}^T_K \cdot (\BinW{\Theta}_G - \BinW{\Theta}_R) \cdot \BinW{\omega}_K + \frac{1}{2} (\BinW{\omega}_K + \BinW{\omega}_R)^T \cdot \BinW{\Theta}_R \cdot (\BinW{\omega}_K + \BinW{\omega}_R)
\end{equation}

\subsubsection{Generalisierte Kraftkomponenten}
In der Untersuchung des 1D-Modelles wurde bereits gezeigt, dass der Würfel ein nicht konservatives System ist, da einerseits über die Motoren mechanische Energie zugeführt wird und andererseits durch Reibung mechanische Energie verloren geht. Deshalb müssen die generalisierten Kraftkomponenten bestimmt werden um mit Hilfe des d'Alembert'schen Prinzip die Bewegungsgleichungen zu ermitteln.
\newline

Wie bereits angesprochen erzeugen die Motoren Momente, welche die Schwungmassen antreiben. Gleichermaßen entsteht in den Lagern der Räder ein Reibmoment welches als linear abhängig von der Winkelgeschwindigkeit modelliert wird. Somit ergibt sich das folgende Moment.

\begin{equation}
\BinW{M}_M = \BinW{T}_M - \bS{C}_{\psi} \cdot \BinW{\dot {\psi}} \hspace{35pt} \bS{C}_{\psi} = \begin{pmatrix}
C_{\psi_1} & 0 & 0 \\
0 & C_{\psi_2} & 0 \\
0 & 0 & C_{\psi_3}
\end{pmatrix}
\end{equation}

Das von der Gravitation verursachte Moment $M_G$ lässt sich aus der aktuellen Position des Schwerpunktes $\bS{r}_C$ und der Schwerkraftvektors $\bS{g}$ berechnen.

\begin{equation}
\BinI{G} = \begin{pmatrix}
0 \\ 0 \\ -m_G \cdot g
\end{pmatrix}
\hspace{35pt}
\BinW{G} = \BinW{D} \cdot \BinI{G} = -m_G \cdot g \cdot \begin{pmatrix}
s_{\varphi_1}s_{\varphi_2} \\ -c_{\varphi_1}s_{\varphi_2} \\ c_{\varphi_2}
\end{pmatrix}
\end{equation}

\begin{equation}
\BinW{M}_G = \BinW{r}_C \times \BinW{G} = -m_G \cdot g \cdot  \begin{pmatrix}
\inW y_C \cdot c_{\varphi_2} + \inW z_c \cdot c_{\varphi_1} s_{\varphi_2} \\
\inW z_C \cdot s_{\varphi_1}s_{\varphi_2} - \inW x_C \cdot c_{\varphi_2} \\
- \inW x_C \cdot c_{\varphi_1}s_{\varphi_2} - \inW y_C \cdot s_{\varphi_1} s_{\varphi_2} 
\end{pmatrix} 
\end{equation}


\newpage
\begin{thebibliography}{\hspace{0.5cm}}
	\bibitem{Cubli1D} Mohanarjah Gajamohan, Michael merz, Igor Thommen, Raffaello D'Andrea: The Cubli: A Cube that can Jump Up and Balance
	\bibitem{Cubli3D_LQR} Mohanarajah Gajamohan, Michael Muehlbach, Tobias Widmer, Raffaello D'Andrea: The Cubli: A Reaction Wheel Based 3D Inverted Pendulum
	\bibitem{Cubli3D_Nonlinear} Michael Muehlbach, Gajamohan Mohanarajah, Raffaello D'Andrea: Nonlinear Analysis and Control of a Reaction Wheel-based 3D Inverted Pendulum
	\bibitem{TheoPhysik1} Wolfgang Nolting: Grundkurs Theoretische Physik 1 - Klassische Mechanik
	\bibitem{TheoPhysik2} Wolfgang Nolting: Grundkurs Theoretische Physik 2 - Analytische Mechanik
	\bibitem{Kane} Thomas R. Kane: Dynamics - Theory and Applications
	\bibitem{PraxisDerDigSigVer} Fernando Puente Le\'on, Sebastian Bauer: Praxis der digitalen Signalverarbeitung
	\bibitem{SimTechLinearUndNichtlinSysteme} Josef Hoffmann, Franz Quint : Simulation technischer linearer und nichtlinearer Systeme mit MATLAB/Simulink
	\bibitem{SystemTheoStochPro} Herbert Schlitt: Systemtheorie für stochastische Prozesse
	\bibitem{SigVer_AnaDigSig} Marin Meyer: Signalverarbeitung - Analoge und digitale Signale, Systeme und Filter
	\bibitem{SuS} Ottmar Beucher: Signale und Systeme - Theorie, Simulation und Anwendung
	\bibitem{RT1} Heinz Unbehauen: Regelungstechnik 1 - Klassische Verfahren zur Analyse und Synthese linearer kontinuierlicher Regelsysteme
	\bibitem{RT2} Heinz Unbehauen: Regelungstechnik 2 - Zustandsregelungen, digitale und nichtlineare Regelsysteme
	\bibitem{lqrnotes} Joao P. Hespanha: Lecture notes on LQR/LQG controller design
	\bibitem{ML_Mitchell} Tom M. Mitchell: Machine Learning
	\bibitem{ML_Bishop} Christopher Bishop: Pattern Recognition and Machine Learning
\end{thebibliography}

\end{document}