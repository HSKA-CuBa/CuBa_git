\section*{Vorwort}
In der folgenden Dokumentation wird das \ac{CuBa}-Projekt vorgestellt, welches im Rahmen eines Entwicklungsprojektes an der Hochschule Karlsruhe durchgeführt wurde. Die Idee für dieses Projekt stammt von dem s.g. Cubli der ETH Zürich. Hierbei handelt es sich um einen Würfel, welcher in der Lage ist selbständig auf seine Ecken und Kanten zu springen und dort zu balancieren. Die Hauptaufgaben dieser Arbeiten bestehen darin einen solchen solchen Würfel zu konstruieren, passende Sensorik bzw. Aktorik auszuwählen, und auf der Basis einer Systemanalyse einen Regelkreis zu entwerfen. 
\newline

Für den Studiengang Mechatronik bildet dieses Projekt einen hervorragenden Rahmen, um die Bereiche der Mechanik, Elektrotechnik und Informatik zu vereinen. Unsere persönliche Erwartung eine interessante und technisch anspruchsvolle Problemstellung zu bearbeiten, welche uns die Möglichkeit unsere fachlichen Kenntnisse und Fähigkeiten zu erweitern.
\newline

An dieser Stelle möchten wir uns ganz herzlich bei unseren Betreuern Herr Prof. Dr. Wietzke und Herr Dipl.-Inf. Münzer bedanken, die uns während des Projektes stehts mit Rat und Tat zur Seite standen. Unser weiterer Dank gilt Linux-Master Christian Steiger, der uns die Linux-Distribution und SW-Treiber zur Verfügung gestellt hat.