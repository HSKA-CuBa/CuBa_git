\subsection{Ausblick}
Auf Grund des straffen Zeitplanes dieses Entwicklungsprojektes mussten einzelne Untersuchungen bzw. Erweiterungen vernachlässigt werden. Deshalb soll dieser Abschnitt einen Ausblick über mögliche Optimierungen schaffen, welche im Rahmen weiterer Projekte untersucht werden können. Eine der größten Herausforderungen hierbei waren die hohen Eigenfrequenzen des Systems die zu einem sehr dynamischen Verhalten führen. Dadurch sind simple Tiefpassfilter ungeeignet um Rauschsignale zu entfernen, da sie zu große Verzögerungen mit sich bringen. Auch der verwendete Regler muss in der Lage sein in kurzer Zeit auf Störungen zu reagieren.

Die verschiedenen Filter wurden in dieser Arbeit zum Teil nur oberflächlich untersucht, besonders der direkte Vergleich der Ansätze beruht lediglich auf dem Vergleich der letztendlichen Reglergüte. In diesem Bereich können weitere Verbesserungen erreicht werden in dem einerseits ein absolutes Referenzsignal gemessen wird um eine Bewertung der Filtersignale durchzuführen. Beispielsweise kann der Winkel $\varphi$ mit einem Drehgeber gemessen werden. Dadurch steht ein weiteres Signal zur Verfügung, das präziser als die Winkelschätzung bzw. die Filter ist und somit als Sollwert betrachtet werden kann. Weiterhin können die Rauschanteile der Sensorsignale näher untersucht werden um eine weitere Verbesserung der Filteralgorithem zu erreichen. Besonders das verwendete Komplementär-Filter verspricht hier großes Verbesserungspotential.

Außerdem wurde im Rahmen dieses Projektes lediglich ein Zustandsregler verwendet. Das Gebiet der Regelungstechnik bietet hier eine Vielzahl verschiedener Ansätze, welche eine zu weiteren Verbesserungen im Systemverhalten führen können. In dieser Hinsicht eignet sich das Projekt auch hervorragend als Testmittel um verschiedene Regler- und Filteralgorithmen miteinander zu vergleichen.