\subsection{Aufspringen}
Das Aufspringen der Würfelseite wird durch das abrupte bremsen der Schwungmasse ermöglicht. Hierbei wird der Drehimpuls der Schwungmasse auf das Gesamtsystem übertragen. Dieser Vorgang kann als nicht elastischer Stoß modelliert werden. Somit ergibt sich aus dem Drehimpulserhaltungssatz folgende Gleichung, wobei $\dot{\varphi}_B$ die Winkelgeschwindigkeit der Würfelseite nach dem Bremsen und $\dot{\psi}_B$ die Winkelgeschwindigkeit der Schwungmasse vor dem Bremsen darstellt.

\begin{equation}
\label{impulserhaltung_bremsen_equation}
(\theta^A_K + \theta^B_R + m_R \cdot l_{AB}) \cdot \dot{\varphi}_B = \theta^B_R \cdot \dot{\psi}_B
\end{equation} 

Um die Würfelseite von der Ruhelage ($\varphi_R = \pm \frac{\pi}{4}$) zu dem Gleichgewichtspunkt ($\varphi_G = 0$) zu bewegen muss Arbeite verrichtet werden. Diese Arbeit $W$ ist gleich der Änderung der potentiellen Energie von der Ruhelage hin zu dem Gleichgewichtspunkt.

\begin{equation}
\label{arbeit_bremsen_equation}
W = V(\varphi_G) - V(\varphi_R) = g(m_K + m_R) l_{AC} \cdot (cos(\varphi_G) - cos(\varphi_R))
\end{equation}

Auf Grund des Energieerhaltungssatzes muss die kinetische Energie des Gesamtsystemes nach dem Bremsvorgang gleich der zu leistenden Arbeit sein um die Würfelseite aufzurichten. Somit kann der Zusammenhang von $\dot{\varphi}_B$ und der Arbeit $W$ wie folgt beschrieben werden.

\begin{equation}
\label{energie_bremsen_equation}
\frac{1}{2}(\theta^A_K + \theta^B_R + m_R \cdot l_{AB}) \dot{\varphi}_B^2 = g(m_K + m_R) l_{AC} \cdot (1 - \frac{1}{\sqrt{2}})
\end{equation}

Mit Hilfe der Gleichungen (\ref{energie_bremsen_equation}) und (\ref{impulserhaltung_bremsen_equation}) kann nun die notwendige Bremsgeschwindigkeit $\dot{\psi}_B$ berechnet werden.

\begin{equation}
\label{psi_bremsen_equation}
\dot{\psi}_B = \sqrt{(2-\sqrt{2}(m_R + m_K) \cdot l_{AC} \cdot g \cdot \frac{\theta^A_K + \theta^B_R + m_R \cdot l_{AB}}{{\theta^B_R}^2}}
\end{equation}

Das obige Modell geht von der Annahme aus, dass es sich um einen perfekt nicht elastischen Stoß handelt und bei der Bewegung der Würfelseite keine Energie verloren geht. Somit besteht eine Abweichung des Modells von den realen Bedingungen. Um diese Abweichungen zu minimieren wird ein, an den Gradientenabstieg angelehnter, Lernalgorithmus implementiert. Nach dem Abbremsen der Schwungmasse werden die Größen $\varphi$ und $\dot{\varphi}$ beobachtet. Tritt ein Nulldurchgang von $\varphi$ auf bedeutet dies, dass die Anfangsgeschwindigkeit $\dot{\varphi}_B$ und somit die Radgeschwindigkeit $\dot{\psi}_B$ zu hoch waren. Tritt jedoch ein Nulldurchgang von $\dot{\varphi}$ auf, folgt, dass $\dot{\varphi}_B$ und $\dot{\psi}_B$ zu niedrig waren. In beiden Fällen kann die Änderung der Energie $\Delta E$, welche nötig ist um den Zielpunkt zu erreichen, berechnet werden.

\begin{equation}
\Delta E = \begin{cases}
\begin{matrix}
(1-cos(\varphi_0))(m_K+m_r)l_{AC} \cdot g  & \hspace{35pt} \vert \hspace{5pt} \dot{\varphi} = 0 \\
-\frac{1}{2}(\theta^A_K + \theta^B_R + m_R \cdot l_{AB}) \cdot \dot{\varphi}^2_0 & \hspace{35pt} \vert \hspace{5pt} {\varphi} = 0 \\
\end{matrix}

\end{cases}
\end{equation}

Mit Hilfe der Drehimpuls- (\ref{impulserhaltung_bremsen_equation}) und Energieerhaltung (\ref{energie_bremsen_equation} wird nun aus der Energieänderung $\Delta E$ die nötige Änderung der Radgeschwindigkeit $\Delta \dot{\psi}_B$ berechnet.

\begin{equation}
\pm \Delta \dot{\psi}_B = \sqrt{2 \cdot \frac{\theta^A_K + \theta^B_R + m_R \cdot l_{AB}}{{\theta^B_R}^2} \cdot \Delta E}
\end{equation}

Die Konvergenz des Lernalgorithmus gegen den Zielwert wird empirisch bewiesen, hierfür ist allerdings das hinzufügen einer Lernrate $\mu$ erforderlich. Daraus ergibt sich letztendlich folgende Vorschrift um den aktuellen Wert der Bremsgeschwindigkeit $\dot{\psi}_B$ zu bestimmen.

\begin{equation}
\dot{\psi}_B := \dot{\psi}_B + \mu \cdot \Delta \dot{\psi}_B \hspace{35pt} \vert \hspace{5pt} 0 < \mu \le 1
\end{equation}